\documentclass[11pt,a4paper]{report}
\usepackage[utf8]{inputenc}
\usepackage[ngerman]{babel}
\usepackage[a4paper,left=3cm,right=3cm,top=3cm,bottom=4cm]{geometry}
\usepackage{listings}
\lstset{
	language=C++,
	keywordstyle=\color{blue},
    stringstyle=\color{red},
    commentstyle=\color{green},
    morecomment=[l][\color{magenta}]
}
\usepackage{hyperref}
\hypersetup{
	colorlinks=true,
	linkcolor=black,
	filecolor=magenta,
	urlcolor=blue,
}

\title{Vulkan - Der Weg zum Dreieck}
\author{Peter Maximilian Kain}
\date{\today}

\begin{document}
\maketitle
\newpage

\tableofcontents
\newpage

\chapter{Einleitung}
Diese Arbeit handelt von Vulkan, der neuesten Grafik-API von der Khronos Group, erschienen im Jahr 2016. Die Khronos Group ist unter anderem bekannt für OpenGL, ebenfalls einer Grafik-API. Vulkan dient nicht als Ersatz für OpenGL, sondern vielmehr als zweite Option. Warum braucht man eine zweite Option?\\
In OpenGL hatte man zunächst den Immediate Mode, mit dem einem der Großteil der Arbeit abgenommen wurde, man ziemlich schnell entwickeln konnte, aber alles unter einem Einbußen an Performance. Performance ist für Grafikanwendungen jedoch ein äußerst wichtiger Faktor, da sie eine Grenze darstellt, die manche Ideen nicht umsetzbar macht. Mit OpenGL 3.3+, auch als "Modern OpenGL" bezeichnet, wurde diese Performancegrenze angehoben. Man hat mehr Kontrolle über Geschehnisse, die vorher alle im Hintergrund passierten und hat so eine höhere Performance.\\
Mit Vulkan wurde diese Grenze noch weiter angehoben, aber da Modern OpenGL noch immer brauchbar, und für Entwickler viel einfacher ist, ist Vulkan, wie der Name schon sagt, ein eigenes Produkt. Vulkan soll den Overhead von Modern OpenGL noch weiter reduzieren und so ressourcenschonender und noch performanter sein. Dafür ist die API äußerst verbos und man muss einen Großteil selbst konfigurieren. Der Vorteil dabei ist, dass man als OpenGL Entwickler noch einmal sieht, was denn alles im Hintergrund für einen gemacht wird.\\
\newpage

\section{Die Vorteile von Vulkan}
Der Hauptkonkurrent zu Vulkan ist momentan Microsoft mit DirectX 12. Um gegen so einen großen Konkurrenten bestehen zu können, braucht es klare Vorteile:\\
Vulkan ist plattformunabhängiger als DirectX 12. DirectX 12 unterstützt Windows 10 und XBox One X, während Vulkan Windows 7, 8 und 10, Linux, MacOS, Android und iOS untertützt. Auch ist Vulkan im Gegensatz zu DirectX 12 quelloffen. Da ich persönlich auch einen OpenGL Hintergrund habe und mein Projekt auf Linux laufen sollte, war meine Wahl einfach.

\section{Die folgenden Kapitel}
Die folgenden Kapitel sollen Meilensteine am Weg zum ersten Dreieck darstellen. In OpenGL ist das erste Dreieck eine Art Hello World, in Vulkan stellt das Dreieck weit mehr Arbeit dar, jedoch hat man zu dem Zeitpunkt die Idee Vulkans verstanden und das langwierige, aber notwendige Setup hinter einem.\\
Jedes Beispiel im src Ordner ist ausführbar und beinhaltet so weit wie möglich nur den Code, den man braucht, um das jeweilige Ziel (Name der Datei) zu erfüllen. Die Vorgehensweise ist dabei hauptsächlich übernommen aus dem \href{https://vulkan-tutorial.com}{Vulkan Tutorial}.

\section{Abhängigkeiten}
Die Beispiele haben folgende Abhängigkeiten: \href{https://www.glfw.org/download.html}{GLFW 3.3.2} und \href{https://vulkan.lunarg.com}{Vulkan}. GLFW3 wird hauptsächlich verwendet, um plattformunabhängig Fenster, und spezifisch für Vulkan, einen Surface zu erstellen. In surface.cpp ist die Vorgehensweise der Erstellung von einem Surface als Kommentar verfasst, aber GLFW3 macht einem die Arbeit leichter.\\
Wozu man Vulkan braucht, sollte selbstverständlich sein. Spezifisch habe ich das Vulkan SDK von LunarG verwendet, weil es empfohlen wird und nicht nur die Notwendigkeiten für Vulkan (Bibliothek und Header), sondern auch Testprogramme und Compiler von GLSL auf SPIR-V (wie später besprochen wird) beinhaltet.

\section{Was wird nicht besprochen}
Diese Arbeit soll keine Dokumentation von Vulkan darstellen (die wäre viel zu umfangreich und im gegebenen Zeitrahmen nicht möglich), sondern vielmehr die Vorgehensweise behandeln, wie man zu den einzelnen Meilensteinen gelangt. Das beinhaltet beispielsweise: Welche Informationen man braucht, um ein Objekt zu erstellen. Jedoch nicht: Eine detaillierte Beschreibung jedes Attributs eines structs, welches man ausfüllen muss, um ein Objekt zu erstellen.

\chapter{Interagieren mit der Grafikkarte}
Mit Vulkan startet man (fast) von Null. Bevor man mit Rendering beginnen kann, muss man zuerst eine Hardware auswählen, welches man zum Rendern verwenden will (physical device). Das Ziel dieses Kapitels ist es, eine Schnittstelle zu dieser Hardware (logical device) zu erstellen. Man müsste für jedes physische Gerät ein logisches Gerät erstellen, aber Multi-GPU Rendering behandelt dieses Kapitel nicht. Da wir auch nicht in tiefere Gebiete, wie Geometry Shaders, vordringen, ist die einzige Vorraussetzung eine Grafikkarte mit Vulkan Support.\\
Die Sections dieses Kapitels beschreiben die Beispiele:
\begin{enumerate}
	\item vulkan\_example
	\item physical\_devices\_and\_queue\_families
	\item logical\_device
\end{enumerate}

\section{Erstellen einer VkInstance}
Eine VkInstance dient als Initialisierungspunkt von Vulkan, bei dem die Applikation Informationen über sich preisgeben kann. Außerdem dient die Instanz dazu, Informationen über physical devices zu bekommen. Fast alle Vulkan Objekte (gekennzeichnet durch Vk, im Gegensatz zu Funktionen vk) werden durch das Übernehmen von Informationen aus einem struct erstellt (Vk Objektname CreateInfo). So muss man, um eine Instanz zu erstellen, das struct VkInstanceCreateInfo ausfüllen. Um dieses struct zu erstellen, braucht man noch das struct VkApplicationInfo.
Fast alle structs haben in Vulkan das Attribut sType. Dieses dient dazu, ein struct im Low-Level Bereich zu identifizieren und ist daher das erste Attribut in so einem struct. Ein sType ist zwar lang, aber klar definiert: VK\_STRUCTURE\_TYPE\_NAME\_OF\_THE\_STRUCTURE\\
Für VkApplicationInfo werden wir also VK\_STRUCTURE\_TYPE\_APPLICATION\_INFO angeben.
Ansonsten enthält VkApplicationInfo Informationen über die Applikation, wie der Name, die Version, und die Version der verwendeten Vulkan API. Auch kann man den Namen und die Version einer Engine angeben, die man verwendet. Anhand dieser Informationen kann der Treiber die Applikation optimieren (bspw. man verwendet eine bekannte Engine).\\
VkInstanceCreateInfo verwendet dieses struct und fügt noch weitere Informationen hinzu, wie die zu verwendeten Validation Layers (besprochen im nächsten Kapitel) und Extensions, die wir benötigen. Vulkan selbst ist wie ein Fundament zu sehen, auf dem man mit Extensions bauen kann. So muss man beispielsweise Extensions hinzufügen, damit Vulkan mit dem Fenster interagieren kann.\\GLFW3 hat dafür die Funktion glfwGetRequiredInstanceExtensions, welche die notwendigen Extensions zurückliefert, die man dann angeben kann.\\
Anschließend kann man mit vkCreateInstance die Instanz erstellen. Vulkan bietet einem dabei die Möglichkeit, Callbacks für einen Allocator anzugeben, wenn man denn einen eigenen verwenden möchte. Wir werden das nicht, und geben für so einen Parameter nullptr an. Außerdem liefert eine vkCreate Funktion typischerweise ein VkResult zurück, welches man auf Werte überprüfen kann. Hat es den Wert VK\_SUCCESS, war das Erstellen erfolgreich, jedoch gibt es noch andere Werte, wie z.B.: VK\_ERROR\_OUT\_OF\_HOST\_MEMORY oder VK\_ERROR\_FEATURE\_NOT\_PRESENT. Letzterer Fehler wird in der nächsten Section mehr Sinn ergeben.

\section{Auswählen eines physical device}
Eine Grafikanwendung hat typischerweise Anforderungen an Grafikkarten, wie zum Beispiel, dass die Grafikkarte ein spezielles Feature, wie Geometry Shaders, unterstützt, oder dass die Grafikkarte aufgrund der Rechenleistung dediziert ist. Die Wahl des richtigen physical devices behandelt diese Themen. Mit vkEnumeratePhysicalDevices bekommt man mithilfe der VkInstance die physical devices im System zurückgeliefert. Danach kann man mit vkGetPhysicalDevice(Properties/Features) bestimmte Properties und Features abfragen, die das physical device hat (Beispiel in Execute()). Man könnte dann die Features und Properties gewichten und das Gerät wählen, dass am besten diese Properties und Features unterstützt (im Idealfall alle gebrauchten).\\
Damit ist man jedoch noch nicht ganz fertig: Man muss auch überprüfen, ob das gewünschte physical device bestimmte Queues hat. Eine VkQueue wird verwendet, um Commands auszuführen (wird später besprochen). Für uns ist momentan wichtig, dass es eine Queue gibt, an die man Commands zum Rendern abgeben kann. Dafür überprüfen wir die Properties der queues auf dem physical device. Diese bekommen wir mithilfe von vkGetPhysicalDeviceQueueFamilyProperties. Die Art der Queue ist im Attribut queueFlags spezifiziert. Wir wollen, dass das VK\_QUEUE\_GRAPHICS\_BIT gesetzt ist, also überprüfen wir das und speichern uns den Index der QueueFamily ab. Nachdem hoffentlich beide Anforderungen erfüllt sind, haben wir unser physical device gewählt! Jetzt müssen wir noch ein logical device erstellen, mit dem Vulkan mit dem physical device interagieren kann.

\section{Erstellen eines logical device}
Bei der Erstellung des VkDevice (logical devices) legt man fest, wieviele Queues man für welche QueueFamilies erstellen will (im Normalfall braucht man nur eine Queue pro QueueFamily), und welche Features und physical device Extensions man verwenden will. Dafür benötigt man die structs VkDeviceQueueCreateInfo und VkDeviceCreateInfo.\\
In VkDeviceQueueCreateInfo gibt man an, wieviele Queues man für welche QueueFamily mit dem VkDevice erstellen will, sowie die Priority der Queues, welche am Ende das Scheduling betrifft. Für jede QueueFamily muss man so ein struct erstellen.\\
In VkDeviceCreateInfo gibt man die Adresse zu den VkDeviceQueueCreateInfo structs an, sowie die Anzahl der VkDeviceQueueCreateInfo structs, wie es in C üblich ist. Wenn es nur ein struct ist, wie in diesem Fall, reicht die Adresse zu diesem und 1 für die Anzahl. Des weiteren braucht man die Adresse zu einem VkPhysicalDeviceFeatures struct, der die Information enthält, ob das Feature aktiv, oder nicht aktiv sein soll. Da wir keine besonderen Features verwenden, erstellen wir einen VkPhysicalDeviceFeatures struct, lassen alles mithilfe der Default-Initialization auf VK\_FALSE und geben die Adresse von diesem an. Zu guter Letzt müssen wir noch die Extensions für das physical device angeben. Da wir für das physical device (noch) keine brauchen, lassen wir einfach den count auf 0.\\
Jetzt können wir mit vkCreateDevice ein logical device erstellen. Die verlangten Queues bekommen wir mit vkGetDeviceQueue. Da geben wir neben dem logical device den Index der QueueFamily an, sowie den Index der Queue (benötigt, wenn man mehr als eine Queue für die QueueFamily angegeben hat, bei uns ist der Index daher 0) und die Adresse einer VkQueue, wo man die Queue speichern will.\\
Nun haben wir ein logical device und können theoretisch schon anfangen, die Swap Chain zu definieren, jedoch handelt das nächste Kapitel erstmal von einer optionalen Möglichkeit zu Debuggen - den Validation Layers.


\chapter{Validation Layers}
Da Vulkan eine sehr verbose API ist, kann sich bei der Menge an Informationen, die man angeben kann, und muss, leicht ein Fehler einschleichen. Validation Layers dienen dazu, unter anderem solche Fehler aufzudecken und stellen quasi Sicherheitsschichten dar, die man durchdringen muss, damit man zur eigentlichen Funktionalität kommt. Klarerweise kann das nur einen Nachteil haben - Performance. Daher aktiviert man sie generell in einem Debug Build und im Release Build kann man sie ganz einfach nicht aktivieren. Der Vorteil dabei ist, dass man, auch wenn man sie aktiviert, man nur die aktivieren kann, die man braucht. Eine Standard Validation Layer ist VK\_LAYER\_KHRONOS\_validation, welche ich im Laufe dieses Guides verwendet habe.\\
Noch mehr Informationen zu Validation Layers gibt es unter \href{https://gpuopen.com/using-the-vulkan-validation-layers/}{GPU Open}\\
Die Section dieses Kapitels beschreibt das Beispiel:
\begin{enumerate}
	\item validation\_layers
\end{enumerate}

\section{Hinzufügen von Validation Layers}
Wie wir schon erfahren haben, werden Validation Layers mit dem Erstellen einer VkInstance aktiviert. Das heißt, wir müssen, wie bei den Extensions die Anzahl und Namen der gewünschten Validation Layers im VkInstanceCreateInfo struct angeben. Dazu müssen wir erst einmal wissen, welche Validation Layers es überhaupt gibt. Mit vkEnumerateInstanceLayerProperties kann man diese Information bekommen. Im Sinne dieses Beispiels gibt es auch noch die Methode CheckIfLayerValid, die überprüft, ob die gewünschte Layer vorhanden ist, bevor sie einem Array hinzugefügt wird. Der Array und deren Größe werden dann der CreateInstance Methode vom ersten Beispiel mitgegeben und mit diesen Informationen sind die gewünschten Validation Layers aktiviert.\\
Will man sich mit diesen ein wenig spielen, enthält die Execute Methode ein kleines Programm, mit dem man verfügbare Layers mit \$PRINT ausgeben kann, Layers hinzufügen kann, und mit \$FINISH zuletzt eine VkInstance mit den angegebenen Layers erstellen kann. Wie schon erwähnt werden wir uns jedoch im weiteren Verlauf mit der Layer VK\_LAYER\_KHRONOS\_validation begnügen.

\chapter{Die Swapchain}
Wer schon Erfahrung mit Grafikprogrammierung hat, der weiß, dass gerenderte Bilder vor der Präsentierung an den Benutzer zuerst in einem Puffer gespeichert werden, wo ein Bild des Puffers nach der Präsentation des angezeigten Bilds getauscht (swap) wird. Der Puffer ist hierfür der Vorteil, dass erstens fertige Bilder präsentiert werden können und nicht On-The-Fly gerendert werden müssen und zweitens, dass der Grafikkarte nicht so schnell langweilig wird. Beispielsweise kann die Grafikkarte nachdem ein Bild gerendert wurde schonmal das nächste Bild rendern, bevor das schon gerenderte Bild überhaupt präsentiert wird.\\
Diese Puffer sind die Swapchain (ein Puffer ist quasi ein Kettenglied), und diese werden wir in diesem Kapitel konfigurieren und erstellen. Doch bevor wir die Swapchain erstellen, müssen wir erstmal, wie in der Einleitung kurz angesprochen, einen Surface erstellen, der Vulkan quasi als Präsentationsoberfläche dient, also als Weg, mit dem Fenster der Anwendung zu kommunizieren.\\
Die Sections dieses Kapitels beschreiben die Beispiele:
\begin{enumerate}
	\item surface
	\item swap\_chain
\end{enumerate}

\section{Erstellen eines Surface}
Mit GLFW3 ist die Erstellung eines VkSurfaceKHR Objekts einfach. Bevor wir diese Erstellung durchgehen, will ich jedoch die wahrscheinliche Frage beantworten, warum bei VkSurfaceKHR am Ende KHR steht. Dieser Anhang kennzeichnet das VkSurfaceKHR Objekt als Teil einer Extension. Welche Extension? Einer der von glfwGetRequiredInstanceExtensions zurückgelieferten, im speziellen: VK\_KHR\_surface. Wir werden auch später bei der Erstellung der Swapchain diesen Anhang bemerken, da die Swapchain ebenfalls Teil der von GLFW3 verlangten Extensions ist. Nun aber zum Erstellen des Surfaces:\\
Das Erstellen eines VkSurfaceKHR Objektes wäre standardmäßig nicht plattformunabhängig. Im Fall von Windows müsste vkCreateWin32SurfaceKHR aufgerufen werden, wenn man X11 verwendet, würde die Funktion dafür vkCreateXlibSurfaceKHR heißen. Die Funktionen verlangen beide unterschiedliche structs, die Informationen über das Fenster brauchen (in Windows HINSTANCE und HWND, z.B.). Da GLFW3 für uns schon das plattformunabhängige Erstellen und Verwalten von Fenstern weg abstrahiert, kann es auch ohne Probleme mit einem einzigen Funktionsaufruf ein VkSurfaceKHR Objekt zurückliefern, nämlich mit glfwCreateWindowSurface (benötigt die Instanz, die Adresse zum GLFWwindow, ein Callback für einen Allocator (= nullptr) und die Adresse des VkSurfaceKHR Objekts, das wir erstellen wollen).\\
Jetzt wäre ein guter Zeitpunkt, den wenigen GLFW3 Code zum Erstellen des GLFWwindow Objektes zu erklären. Der Code dafür ist schon im Beispiel vulkan\_example zu finden und besteht im Ganzen aus 3 Zeilen:
\begin{lstlisting}[language=C++]
glfwWindowHint(GLFW_CLIENT_API, GLFW_NO_API);
glfwWindowHint(GLFW_RESIZABLE, GLFW_FALSE);
glfwWindow = glfwCreateWindow(800, 600, name.c_str(),nullptr,nullptr);
\end{lstlisting}
Auf die erste und zweite Zeile kommen wir später noch zurück. Die dritte Zeile betrifft das Erstellen eines GLFWwindow Objekts. Die wichtigen Parameter sind hierfür Breite/Höhe des Fensters, sowie der Titel. Der loop zum Offenhalten des Fensters sieht folgendermaßen aus:
\begin{lstlisting}[language=C++]
while (!glfwWindowShouldClose(glfwWindow)) {
	glfwPollEvents();
}
\end{lstlisting}
Der Code ist ziemlich selbsterklärend. Solange das Fenster nicht geschlossen werden soll (also kein Close-Event), dann halte nach Events ausschau (wie z.B. einem Close-Event). Normalerweise müsste man auch glfwSwapBuffers aufrufen und das ist der Zeitpunkt, wo wir die ersten beiden Zeilen von oben besprechen können: Die erste Zeile macht das Aufrufen von glfwSwapBuffers zu einem Fehler. Wir erstellen ja mit Vulkan unsere eigene Swapchain, die das Swappen von Image Buffers übernimmt. Die zweite Zeile verhindert einfach das Ändern der Größe des Fensters. Wenn das Fenster, und somit der Surface verändert werden würde, müsste man nämlich die Swapchain neu erstellen, da die derzeit gespeicherten Informationen ungültig wären (dazu später mehr).\\
Nun, da wir den GLFW3 Code besprochen hätten, zurück zur Erstellung des Surfaces - wir sind nämlich noch nicht fertig. Unsere Anforderungen an das physical device haben sich geändert! Das physical device muss erst einmal auch Surfaces unterstützen! Also passen wir den Code für die Erstellung von physical und logical devices noch einmal an und überprüfen den Support für Surfaces in den QueueFamilies. Es sollte nämlich eine Queue geben, die letztendlich mit unserer Swapchain interagiert und ihr die gerenderten Bilder zukommen lässt. Das können wir mit der Funktion vkGetPhysicalDeviceSurfaceSupportKHR überprüfen. Diese erwartet sich das physical device, den QueueFamilyIndex, unser VkSurfaceKHR Objekt und einen VkBool32 output, der das Ergebnis beinhaltet.\\
Diesen Code können wir zu dem hinzufügen, der überprüft, ob das jeweilige physical device eine rendering Queue hat, also zu den physical device checks. Beim Erstellen des logical devices müssen wir auch beachten, dass es nun eine zweite QueueFamily gibt und noch ein VkDeviceQueueCreateInfo struct für diese QueueFamily hinzufügen (wir wollen wieder nur eine Queue erstellen). Klarerweise müssen wir auch den VkDeviceCreateInfo struct anpassen, da wir zwei VkDeviceQueueCreateInfo structs haben, die wir angeben müssen. Zu guter Letzt noch ein Aufruf von vkGetDeviceQueue um die neue Queue zu speichern und - das wars!

\newpage
\section{Erstellen der Swapchain}
Um überhaupt eine Swapchain erstellen zu können, müssen wir erstmal eine weitere Extension hinzufügen. Warum ist diese Extension nicht Teil der von GLFW3 verlangten Extensions? Ich vermute mal, weil die Erstellung der Swapchain nicht mehr im Anwendungsgebiet von GLFW3 liegt. Und in Vulkan ist das Konzept "You get what you pay for" ja allgegenwärtig.\\
Aber wie hieß noch schnell die Extension genau? Da kann uns Vulkan aushelfen, und zwar ist ein Makro VK\_KHR\_SWAPCHAIN\_EXTENSION\_NAME definiert. Wir müssen also nur wissen, dass VkSwapchainKHR KHR als Anhang hat und Swapchain heißt. Das bringt auf ersten Blick nicht viel, wenn man weiß, dass die Extension "VK\_KHR\_swapchain" heißt, aber immerhin kann man sich mit dem Makro sicher sein, dass man es richtig geschrieben hat. Wie fügen wir jetzt die Extension hinzu? Mit dem Erstellen eines logical devices natürlich! Wir sollten vorher noch sicherstellen, dass das physical device diese Extension unterstützt, also wieder den Code modifizieren, der die physical devices überprüft und schauen, dass alle benötigten Extensions in den verfügbaren Extensions vorkommen. Wie bekommt man nochmal alle verfügbaren Extensions? Mit vkEnumerateDeviceExtensionProperties.\\
Nun, da wir diesen Schritt erledigt haben, können wir die Swapchain erstellen. Jedoch brauchen wir noch Informationen: Wir wissen ja nicht, welche Optionen zum Konfigurieren der Swapchain überhaupt unterstützt werden! Wir müssen 3 Dinge überprüfen: Was unser Surface so unterstützt - VkSurfaceCapabilitiesKHR, dann welches Farbformat und Farbbereich unterstützt wird - VkSurfaceFormatKHR, und zuletzt, wie Bilder in der Swapchain angezeigt werden können - VkPresentModeKHR. Diese Informationen lassen sich mithilfe von vkGetPhysicalDeviceSurface(Capabilities/Formats/PresentModes)KHR bekommen und sind bei mir, wie auch im \href{https://vulkan-tutorial.com}{Vulkan Tutorial}, zusammen in einem struct gespeichert. Nun müssen wir aus den verfügbaren Optionen wählen, also die Swapchain konfigurieren. Dazu braucht man unter anderem die 3 Objekte VkSurfaceFormatKHR, VkPresentModeKHR und VkExtent2D.\\
Vulkan lässt einem sehr viele Optionen, aber normalerweise kann man sich an ein paar festhalten: Beispielsweise sind 8-Bit RGBA Kanäle, sowie das SRGB Format nicht verkehrt. Also gehen wir, wie bei fast jeder Konfiguration, alle verfügbaren Optionen (VkSurfaceFormatKHR) durch und wählen uns ein VkSurfaceFormatKHR Objekt aus, wo das .format Attribut den Wert VK\_FORMAT\_B8G8R8A8\_SRGB, und das .colorSpace Attribut den Wert VK\_COLOR\_SPACE\_SRGB\_NONLINEAR\_KHR hat. Wenn uns diese zwei Dinge komplett egal sind, oder es diese Optionen nicht gibt, kann man immer das Format so konfigurieren, dass man einfach das erste verfügbare Format nimmt.\\
Haben wir uns das gewünschte VkSurfaceFormatKHR Objekt gespeichert, können wir zum Beispiel mit VkPresentModeKHR weitermachen. Mit diesem Objekt wird bestimmt, wann ein Bild dem User Präsentiert wird. Garantiert verfügbar ist der Modus VK\_PRESENT\_MODE\_FIFO\_KHR, der aussagt, dass die Swapchain eine Queue ist. Die Bilder kommen vorn auf den Bildschirm und werden von hinten von der Grafikkarte nachgefüllt. Ist diese Queue voll, muss die Grafikkarte warten. Das ist natürlich nicht optimal, da wir ja unsere Rechenleistung bestmöglich ausnützen wollen. Deshalb überprüfen wir, ob es nicht den PresentMode VK\_PRESENT\_MODE\_MAILBOX\_KHR gibt. Da müsste die Grafikkarte nicht mehr warten, sondern überschreibt einfach die Bilder in der Queue mit neu gerenderten. Wenn es diesen Modus nicht gibt, geben wir uns aber auch mit dem ersten zufrieden.\\
Zu guter Letzt brauchen wir noch ein Objekt: Ein VkExtent2D Objekt, das in dem Fall als Auflösung der Bilder in der Swapchain dient. Dazu brauchen wir das VkSurfaceCapabilitiesKHR Objekt. Dieses Objekt beinhaltet folgende Informationen über den Extent: .(max/min)ImageExtent und .currentExtent. Generell speichert man sich einfach den .currentExtent, aber es kann sein, dass im .currentExtent die Dimensionen den Maximalwert von einem 32-Bit unsigned Integer annehmen. In diesem Fall müssen (und können) wir den Extent selbst bestimmen, werden dabei aber durch .(max/min)ImageExtent beschränkt. Wir lösen das so, dass wir uns von GLFW3 mithilfe von glfwGetWindowSize die Größe des Fensters mitteilen lassen und sicherstellen, dass die Breite und Höhe nicht über .(max/min)ImageExtent hinausragt. So können wir selbst ein VkExtent2D Objekt erstellen, indem wir .width und .height setzen und dieses abspeichern.\\
Jetzt heißt es wieder: Ein struct mit diesen Informationen ausfüllen. In dem Fall benötigen wir das struct VkSwapchainCreateInfoKHR. Dieses struct ist sehr groß und benötigt viele Informationen. Um das struct ganz ausfüllen zu können müssen wir uns noch Gedanken darüber machen, wieviele Bilder (=Puffer) wir denn in unserer Swapchain haben wollen. Das VkSurfaceCapabilitiesKHR beinhaltet jedenfalls schon einmal einen .(max/min)ImageCount. Wir könnten uns mit dem .minImageCount zufrieden geben, aber generell wird empfohlen, mindestens ein Bild mehr zu haben als der .minImageCount. Allerdings dürfen wir nicht mehr als der .maxImageCount verlangen. Ist dieser gleich 0 ist jedoch kein Maximum vorgegeben.\\
Nachdem wir dies sichergestellt haben, können wir anfangen, VkSwapchainCreateInfoKHR zu befüllen. Neben unserem VkSurfaceKHR Objekt und den ausgesuchten PresentModes usw. benötigt der struct weitere Informationen, wie die Schichten eines Bildes, wenn wir beispielsweise 3D Bilder rendern wollen, oder die imageUsage. Man kann direkt in die Swapchain rendern (in dem Fall auf VK\_IMAGE\_USAGE\_COLOR\_ATTACHMENT\_BIT setzen), oder, wenn man das Bild nach dem Rendern bearbeiten möchte (Post Processing), dieses später hinzufügen (VK\_IMAGE\_USAGE\_TRANSFER\_DST\_BIT).\\
Wenn unsere QueueFamilies für Graphics und Presentation nicht die gleichen sind, müssen wir den Zugriff auf Bilder der Swapchain definieren. Wir wollen ja mit einer Queue die Bilder rendern und mit einer anderen Queue die Bilder der Swapchain zukommen lassen. Haben diese beiden Queues nicht dieselbe QueueFamily, kommt das zu Konflikten. Deshalb kann man zwischen VK\_SHARING\_MODE\_(CONCURRENT/EXCLUSIVE) wählen, sowie die Anzahl der queueFamilies und deren Indices angeben. Schließlich kann man, wenn man möchte, noch Transformationen vor dem Anzeigen angeben, sowie angeben, ob das Fenster anhand des Alphakanals durchsichtig sein soll (wie man zum Beispiel beim Linux Terminal einstellen kann), und, ob die Pixel, die sich beispielsweise hinter einem anderen Fenster befinden, uns egal sind, oder nicht (ersteres führt klarerweise zu besserer Performance).\\
Jetzt bleibt uns nur noch ein Feld übrig: .oldSwapchain. Ich habe schonmal erwähnt, dass die Swapchain z.B. beim Vergrößern des Fensters z.B. ungültig werden kann. In dem Fall muss man sie neu erstellen, und die alte Swapchain bei dem Feld angeben. Da wir unsere erste Swapchain erstellen, befassen wir uns nicht damit und geben VK\_NULL\_HANDLE an.\\
Fast zwei Seiten später können wir endlich ein VkSwapchainKHR Objekt mittels vkCreateSwapchainKHR erstellen, und die Bilder der Swapchain für das spätere Rendering mittels vkGetSwapchainImagesKHR bekommen.

\chapter{Frame- und Commandbuffers}
\chapter{Die Grafikpipeline}
\chapter{Das Dreieck}


\end{document}